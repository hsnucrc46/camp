\documentclass{article}
\usepackage{../settings}

\begin{document}
\hexcover{Python 實際應用—Line Bot}{電算社}{作者:魏采萱}{SpringGreen}{第1天}

\begin{large}

\section{什麼是 PYPI 模組?}
    \begin{mintbox}{import 範例}{SpringGreen}{python}
        % TODO: 不太確定是不是這個
        import linebot
    \end{mintbox}
\section{LineBot 是什麼}
\section{LineBot 應用}
\section{LineBot 基礎語法}
\begin{boxpar}{SpringGreen}{範例程式}
首先先來看看範例
    \begin{mintbox}{原始碼}{SpringGreen}{python} % TODO: 請更改以下程式碼,此為 AI 生成
    from linebot import LineBot, WebhookHandler
    from linebot.exceptions import InvalidSignatureError
    from linebot.models import MessageEvent, TextMessage, TextSendMessage

    # Line Bot API credentials
    LINE_CHANNEL_ACCESS_TOKEN = 'YOUR_CHANNEL_ACCESS_TOKEN'
    LINE_CHANNEL_SECRET = 'YOUR_CHANNEL_SECRET'

    # Create a Line Bot instance
    bot = LineBot(LINE_CHANNEL_ACCESS_TOKEN)

    # Create a WebhookHandler instance
    handler = WebhookHandler(LINE_CHANNEL_SECRET)

    # Define a function to handle incoming messages
    @handler.add(MessageEvent, message=TextMessage)
    def handle_message(event):
        # Get the user's message
        msg = event.message.text

        # Respond with a "Hello World!" message
        bot.reply_message(event.reply_token, TextSendMessage(text='Hello World!'))

    # Define a function to handle invalid signature errors
    @handler.error
    def error_handler(error):
        print(f'Error: {error}')

    # Run the bot
    if __name__ == '__main__':
        import os
        port = int(os.environ.get('PORT', 5000))
        app.run(host='0.0.0.0', port=port)
    \end{mintbox}
\end{boxpar}

\end{large}
\end{document}
