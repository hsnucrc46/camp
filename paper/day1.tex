\documentclass{article}
\usepackage{../settings}

\begin{document}
\hexcover{Python 基礎教學}{讓你能從零開始}{作者:曾嘉禾}{Dandelion}{第1天}

\begin{large}

%\begin{olivertzeng}
\section{python 是什麼}
    \begin{quote}
        Python(英國發音:/ˈpaɪθən/;美國發音:/ˈpaɪθɑːn/),是一種廣泛使用的解釋型、高級和通用的編程語言。Python支持多種編程范型,包括結構化、過程式、反射式、面向對象和函數式編程。它擁有動態類型系統和垃圾回收功能,能夠自動管理內存使用,並且其本身擁有一個巨大而廣泛的標准庫。它的語言結構以及面向對象的方法,旨在幫助程序員為小型的和大型的項目編寫邏輯清晰的代碼。
        \source{--- 維基百科}
    \end{quote}
    這有點無聊,再試一次吧。Python 是一種用來讓你看似與電影裡的主角一般,正在看著黑黑的畫面寫出能
    DDoS FBI 官網的電神,但實際上你只是利用簡單的程式知識罷了。

    \subsection{編譯 vs 直譯}
    \begin{tabular}{|c|c|c|}
\hline
& 編譯(Compile) & 直譯(Interpret) \\
程式語言例子 & C++ & Python \\
        執行過程 & 將原始碼轉換為二進位執行檔(binary),然後執行 &
        將原始碼逐行翻譯成二進位後執行 \\
執行速度 & 快 & 慢 \\
錯誤檢查 & 編譯時檢查錯誤 & 執行時檢查錯誤 \\
\hline
\end{tabular}
    \subsection{電腦如何判斷陰陽頓挫}
    當你在讀文章時,你會怎麼判斷一個句子的結束?對你而言是句點。驚嘆號!問號?等標點符號判斷句子的結束。但對英文使用者而言是這種句點「.」而不是「。」
    在程式語言內,電腦要判斷每行程式(如同日常生活中的語句)也是需要標點符號的標記。如 C
    語言系列以及 javascript 是用分號 ; 以及括號對 {[()]} 當作判斷依據,而 python
    是使用換行以及縮排(indent)來判斷。換句話說,在 python
    內,雖然語法較其他語言簡單,但是比起其他語言,縮排的重要性以及統一性高出許多。

\section{python 基礎語法}
\section{Hello, World!}
\begin{mintbox}{範例程式}{Dandelion}{python}
print("Hello, World!")
\end{mintbox}

\begin{boxpar}{Dandelion}{先了解程式的意思吧}
    你可能之前就有聽說過,但在 python 內 print
    不是指列印,而是指在螢幕上顯示出你所指定的某些文字
\end{boxpar}
\section{資料型態}
\begin{colbox}{Dandelion}{Python 資料型態表格}
\begin{table}[]
\begin{tabular}{lccc}
    \hline
\multicolumn{1}{c}{資料型態} & 中文名稱 & 英文名稱      & Bytes  \\
\hline
int                      & 整數   & integer   & 4      \\
str                      & 字串   & string    & 依照字串長度 \\
char                     & 字元   & character & 1      \\
float                    & 浮點數 & float     & 8 \\
bool                     & 布林   & boolean   & 1 \\
list                     & 列表   & list      & 依照列表長度 \\
\hline
\end{tabular}
\end{table}
\end{colbox}
\begin{boxpar}{Dandelion}{變數與常數}
    在電腦中要如何儲存資料到記憶體呢?就像搬家一樣,儲存資料需要有容器裝著才好分類管理,所以變數與常數就是資料的容器,隨時都可以拿出來取用。
    \begin{itemize}
       \item \textbf{變數:}在執行過程中可以改變內容物的容器
       \item \textbf{常數:}一旦存進容器後只能聰當標本,在執行中無法更改(例如 π)
    \end{itemize}
\end{boxpar}

%\end{olivertzeng}

%\begin{hi}
    \section{條件判斷}
    \section{迴圈}
    \section{函式}
    \section{類別}
%\end{hi}
\end{large}
\end{document}
