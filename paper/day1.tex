\documentclass{article}
\usepackage{../settings}

\begin{document}
\hexcover{Python 基礎教學}{讓你能從零開始}{作者:曾嘉禾}{Dandelion}{第1天}

\begin{large}

%\begin{olivertzeng}
\section{python 是什麼}
    \begin{quote}
        Python(英國發音:/ˈpaɪθən/;美國發音:/ˈpaɪθɑːn/),是一種廣泛使用的解釋型、高級和通用的編程語言。Python支持多種編程范型,包括結構化、過程式、反射式、面向對象和函數式編程。它擁有動態類型系統和垃圾回收功能,能夠自動管理內存使用,並且其本身擁有一個巨大而廣泛的標准庫。它的語言結構以及面向對象的方法,旨在幫助程序員為小型的和大型的項目編寫邏輯清晰的代碼。
        \source{--- 維基百科}
    \end{quote}
    這有點無聊,再試一次吧。Python 是一種用來讓你看似與電影裡的主角一般,正在看著黑黑的畫面寫出能
    DDoS ㄈㄅㄌ(FBI)網站的電神,但實際上你只是利用簡單的程式知識而已。
\section{python 應用}
%\end{olivertzeng}

%\begin{hi}
\section{python 基礎語法}
\section{Hello, World!}
\begin{mintbox}{範例程式}{Dandelion}{python}
print("Hello, World!")
\end{mintbox}

\begin{boxpar}{Dandelion}{先了解程式的意思吧}
    你可能之前就有聽說過,但在 python 內 print 不是指列印,而是指在螢幕上顯示出你所指定的某些文字
\end{boxpar}

%\end{hi}
\end{large}
\end{document}
