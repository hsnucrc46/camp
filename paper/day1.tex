\documentclass{article}
\usepackage{../settings}

\begin{document}
\hexcover{Python 基礎教學}{讓你能從零開始}{作者:曾嘉禾}{Dandelion}{第1天}

\begin{large}

%\begin{olivertzeng}
\section{python 是什麼}
    \begin{quote}
        Python(英國發音:/ˈpaɪθən/;美國發音:/ˈpaɪθɑːn/),是一種廣泛使用的解釋型、高級和通用的編程語言。Python支持多種編程范型,包括結構化、過程式、反射式、面向對象和函數式編程。它擁有動態類型系統和垃圾回收功能,能夠自動管理內存使用,並且其本身擁有一個巨大而廣泛的標准庫。它的語言結構以及面向對象的方法,旨在幫助程序員為小型的和大型的項目編寫邏輯清晰的代碼。
        \source{--- 維基百科}
    \end{quote}
    這有點無聊,再試一次吧。Python 是一種用來讓你看似與電影裡的主角一般,正在看著黑黑的畫面寫出能
    DDoS FBI 官網的電神,但實際上你只是利用簡單的程式知識罷了。

    \subsection{編譯 vs 直譯}
    \begin{tabular}{|c|c|c|}
\hline
& 編譯(Compile) & 直譯(Interpret) \\
程式語言例子 & C++ & Python \\
        執行過程 & 將原始碼轉換為二進位執行檔(binary),然後執行 &
        將原始碼逐行翻譯成二進位後執行 \\
執行速度 & 快 & 慢 \\
錯誤檢查 & 編譯時檢查錯誤 & 執行時檢查錯誤 \\
\hline
\end{tabular}
    \subsection{電腦如何判斷陰陽頓挫}
    當你在讀文章時,你會怎麼判斷一個句子的結束?對你而言是句點。驚嘆號!問號?等標點符號判斷句子的結束。但對英文使用者而言是這種句點「.」而不是「。」
    在程式語言內,電腦要判斷每行程式(如同日常生活中的語句)也是需要標點符號的標記。如 C
    語言系列以及 javascript 是用分號 ; 以及括號對 {[()]} 當作判斷依據,而 python
    是使用換行以及縮排(indent)來判斷。換句話說,在 python
    內,雖然語法較其他語言簡單,但是比起其他語言,縮排的重要性以及統一性高出許多。

\section{python 基礎語法}
\section{Hello, World!}
\begin{mintbox}{範例程式}{Dandelion}{python}
print("Hello, World!")
\end{mintbox}

\begin{boxpar}{Dandelion}{先了解程式的意思吧}
    你可能之前就有聽說過,但在 python 內 print
    不是指列印,而是指在螢幕上顯示出你所指定的某些文字
\end{boxpar}
\section{資料型態}
\begin{colbox}{Dandelion}{Python 資料型態表格}
\begin{table}[]
\begin{tabular}{lccc}
    \hline
\multicolumn{1}{c}{資料型態} & 中文名稱 & 英文名稱      & Bytes  \\
\hline
int                      & 整數   & integer   & 4      \\
str                      & 字串   & string    & 依照字串長度 \\
char                     & 字元   & character & 1      \\
float                    & 浮點數 & float     & 8 \\
bool                     & 布林   & boolean   & 1 \\
list                     & 列表   & list      & 依照列表長度 \\
\hline
\end{tabular}
\end{table}
\end{colbox}
\begin{boxpar}{Dandelion}{變數與常數}
    在電腦中要如何儲存資料到記憶體呢?就像搬家一樣,儲存資料需要有容器裝著才好分類管理,所以變數與常數就是資料的容器,隨時都可以拿出來取用。
    \begin{itemize}
       \item \textbf{變數:}在執行過程中可以改變內容物的容器
       \item \textbf{常數:}一旦存進容器後只能聰當標本,在執行中無法更改(例如 π)
    \end{itemize}
\end{boxpar}

%\end{olivertzeng}

%\begin{hi}\section{變數}

    \begin{LARGE}
        「變數」是什麼?\\
        \end{LARGE}
    
        變數是程式中很重要的一部份。簡單來說,就是「內容可以改變」的數,可以做為資料的容器。任何的內容型態,像是數字、文字、甚至是一串資料,都可以裝在變數裡面。\\
    
        變數常出現在數學公式,例如位置、長度就是數字類的變數。\\
    
    
    
        \subsection{資料型態}
    
        資料有很多種類型,不同種可能是可以裝不同的東西,例如數字與字元;或是可以容納的大小不同,可以減少需要的資源。\\
        \\
        \begin{LARGE}
        為什麼要區分資料型態?\\
        \end{LARGE}
        \\
    
        區分出不同的資料類別可以讓電腦在運算的時候,知道它裡面的資料是什麼。\\
        這可以確保在需要數字的時候不會跑出字元,\\
        不同的資料在使用同意種運算符號的時候,可能會有不同的結果\\
    
    
    
        \subsection{變數名稱}
        變數的取名是非常重要的,如果隨便取名的話,不只是其他人會看不懂,甚至自己過個幾天也會忘記這個變數代表什麼。\\
        \\
        \begin{LARGE}
        變數的命名\\
        \end{LARGE}
        \\
        通常,如果
    
    %\end{olivertzeng}
    
    %\begin{hi}
    
    \section{運算子}
        運算子可以對資料和變數進行操作,\\
        \\
        \begin{LARGE}
        算數運算子\\
        \end{LARGE}
        \\
        算數運算子包含了常見的數學運算,例如:加法、減法、乘法、除法、次方、取餘數、取商數。\\
        \begin{tabular}{ |c|c|c|c| } 
            \hline
            名稱 & 運算子 & 例子 & 結果 \\
            加法 & + & x = 5 + 3 & x = 8 \\
            減法 & - & x = 5 - 3 & x = 2 \\
            乘法 & * & x = 5 * 3 & x = 15 \\
            除法 & / & x = 5 / 3 & x = 1.66... \\
            次方 & ** & x = 5 ** 3 & x = 125 \\
            取餘數& \% & x = 5 \% 3 & x = 2 \\
            取整除& // & x = 5 // 3 & x = 1 \\
            \hline
        \end{tabular}
        \\
        \begin{LARGE}
        賦予運算子\\
        \end{LARGE}
        \\
        賦予運算子可以改變在它左邊的變數的內容,全部都有等號,是從其他運算子衍伸而來。\\
        \\
        \begin{tabular}{ |c|c|c|c| } 
            \hline
            名稱 & 運算子 & 例子(x=7) & 結果 \\
            加法 & += & x += 2 & x = 9 \\
            減法 & -= & x -= 2 & x = 5 \\
            乘法 & *= & x *= 2 & x = 14 \\
            除法 & /= & x /= 2 & x = 3.5 \\
            次方 & **= & x **= 2 & x = 49 \\
            取餘數& \%= & x \%= 2 & x = 1 \\
            取整除& //= & x //= 2 & x = 3 \\
            \hline
        \end{tabular}
        \\
        除了這幾個以外,還有一些賦予運算子會在之後提到。
    
    
    \section{邏輯判斷}
    
        邏輯判斷
    
        \subsection{比較運算子}
    
        比較運算子可以比較兩個數字的大小,然後輸出是不是符合運算子的關係。
        比較運算子包含了數學裡面的等號、各種不等號。\\
        \\
        \begin{tabular}{ |c|c|c|c| } 
            \hline
            名稱 & 運算子 & 例子(x=7) & 結果 \\
            等於 & == & x  2 &  \\
            不等於 & != & x  2 &  \\
            大於 & > & x  2 &  \\
            小於 & < & x  2 &  \\
            大於或等於 & >= & x  2 &  \\
            小於或等於 & <= & x  2 &  \\
            \hline
        \end{tabular}
    
    
    
        \subsection{邏輯運算子}
    
        邏輯運算子可以把
    
    \section{迴圈}
    
        有時候,我們會需要重複執行一些一樣或非常相似的程式碼,可以使用迴圈來執行。
        使用迴圈不只可以寫得更快,也能讓程式更加簡潔易懂,便於維護。
        以下是兩種迴圈的介紹:
    
        \subsection{for迴圈}
    
        for迴圈通常是用在知道一個範圍的情況下,\\
        用法是:\\
        \begin{lstlisting}
            for 變數 in 一個範圍/一個可以的物件:
                一段程式碼
        \end{lstlisting}
    
    
    
        \subsection{while迴圈}
    
    
        \subsection{退出迴圈}
    
    
    \section{函式}
    
    
        \subsection{}
    
    
    \section{類別}
    
    
        \subsection{}
    
    
%\end{hi}
\end{large}
\end{document}
