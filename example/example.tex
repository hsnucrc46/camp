\documentclass{article}
\usepackage{../settings}

\begin{document}
\hexcover{範例 \LaTeX}{展示 settings.sty 內的函式}{作者:曾嘉禾}{Emerald}{電算社}

\begin{large}
\begin{boxpar}{Emerald}{一段文章}
早上好中國
\begin{mintbox}{程式上色(花了超久才修好)}{Emerald}{python}
print("Hello, World!")
usr = int(input())
if usr >= 46:
    print("Younger")
else:
    print("Older")
\end{mintbox}
\begin{mintbox}{C++}{Emerald}{cpp}
#include <bits/stdc++.h>
using namespace std;

int main(int argc, char *argv[]) {
	int buf = 2;
	while (buf % 71) {
		buf *= 2;
		buf++;
	}
	if (buf % 3) {
		cout << "turn left\n";
	} else {
		cout << "turn right\n";
	}
	return 0;
}

\end{mintbox}
\end{boxpar}
\begin{tcolorbox}[title=多選題,colback=Emerald!5!white,colframe=Emerald!75!black]

下列哪一個有關世界情勢的說明何者政確?\\

\begin{enumerate}[(A)]
    \item 日本是最大產油國
    \item 地球是方的
    \item 美國民主外送很民主
    \item 地球是平的
    \item 中華民國已經拿回秋海棠
\end{enumerate}

\subsection*{簡答}
答案:(A)、(C)、(E)\\

\subsection*{詳解}

(B)錯誤原因:因為 D 是錯的\\
(D)錯誤原因:因為 B 是錯的\\
\end{tcolorbox}
\end{large}
來源: \href{https://github.com/hsnucrc46/camp}{camp}
\end{document}
